%%%%%%%%%%%%%%%%%%%%%%%%%%%%%%%%%%%%%%%%%
%
% Igor Duarte Cardoso
% http://igordcard.com
%
% Official Curriculum Vitae,
% available at:
% https://github.com/igordcard/cv
%
% History and version tracking
% deferred to the git repository.
%
% All remaining credits are
% provided in the next block.
%
%%%%%%%%%%%%%%%%%%%%%%%%%%%%%%%%%%%%%%%%%

%%%%%%%%%%%%%%%%%%%%%%%%%%%%%%%%%%%%%%%%%
% Plasmati Graduate CV
% LaTeX Template
% Version 1.0 (24/3/13)
%
% This template has been downloaded from:
% http://www.LaTeXTemplates.com
%
% Original author:
% Alessandro Plasmati (alessandro.plasmati@gmail.com)
%
% License:
% CC BY-NC-SA 3.0 (http://creativecommons.org/licenses/by-nc-sa/3.0/)
%
% Important note:
% This template needs to be compiled with XeLaTeX.
% The main document font is called Fontin and can be downloaded for free
% from here: http://www.exljbris.com/fontin.html
%
%%%%%%%%%%%%%%%%%%%%%%%%%%%%%%%%%%%%%%%%%

%----------------------------------------------------
%	PACKAGES AND OTHER DOCUMENT CONFIGURATIONS
%----------------------------------------------------

\documentclass[letter,10pt]{article} % Default font size and paper size\usepackage{fullpage}

\usepackage{fontspec} % For loading fonts
\defaultfontfeatures{Mapping=tex-text}
\setmainfont[UprightFont = * Regular]{Fontin} % Main document font
\usepackage[top=0.75in, bottom=0.75in, left=1.0in, right=1.0in]{geometry}

\usepackage{xunicode,xltxtra,url,parskip} % Formatting packages
%\usepackage{url,parskip} % Formatting packages

\usepackage[usenames,dvipsnames]{xcolor} % Required for specifying custom colors

\usepackage{hyperref} % Required for adding links	and customizing them
\definecolor{linkcolour}{rgb}{0,0.2,0.6} % Link color
\hypersetup{colorlinks,breaklinks,urlcolor=linkcolour,linkcolor=linkcolour} % Set link colors throughout the document

\usepackage{titlesec} % Used to customize the \section command
\titleformat{\section}{\Large\scshape\raggedright}{}{0em}{}[\titlerule] % Text formatting of sections
\titlespacing{\section}{0pt}{3pt}{3pt} % Spacing around sections

\begin{document}

\pagestyle{empty} % Removes page numbering

\font\fb=''[cmr10]'' % Change the font of the \LaTeX command under the skills section


%----------------------------------------------------
%	NAME AND CONTACT INFORMATION
%----------------------------------------------------

\par{\centering{\Huge Igor Duarte \textsc{Cardoso}}\bigskip\par} % Your name

\section{Basic Data}

\begin{tabular}{rl}
\textsc{Degree:} & Master of Science in Computer and Telematics Engineering\\
\textsc{Email:} & \href{mailto:igordcard+cv@gmail.com}{igordcard@gmail.com}\\
%\textsc{Web:} & \href{http://igordcard.com}{igordcard.com}\\
\textsc{GitHub:} & \href{http://github.igordcard.com}{github.igordcard.com}\\
\textsc{LinkedIn:} & \href{http://linkedin.igordcard.com}{linkedin.igordcard.com}\\
%\textsc{GitHub:} & \href{https://github.com/igordcard}{github.com/igordcard}\\
%\textsc{LinkedIn:} & \href{https://www.linkedin.com/in/igordcard}{linkedin.com/in/igordcard}\\
\end{tabular} \\

%----------------------------------------------------
%	OBJECTIVES AND MOTIVATIONS
%----------------------------------------------------

\section{Objectives}

My short-term objectives are currently to keep working on the fields of Network Virtualization and Cloud Computing (IaaS) against projects like OpenStack, as a Software Engineer, preferably doing research.

I want to take part in the evolution of existing (and development of new) telecommunication technologies 
and core systems, learning and improving day after day. \\


%----------------------------------------------------
%	EDUCATION
%----------------------------------------------------

\section{Education}

\begin{tabular}{rl}	
\textsc{2014} & Master of Science in \textsc{Computer and Telematics Engineering}, \\
&\textbf{Universidade de Aveiro}, Aveiro, Portugal, \\
& Dissertation: ``Network Infrastructure Control for Virtual Campus'', \\
& Integrated Master's includes both Bachelor's and Master's degrees: \\
& Degree with a broad Computer Science foundation and multiple practical projects.\\
& \textit{Final grade: 16 out of 20.}\\
&\\

\textsc{2008} & High School, \\
&\textbf{Agrupamento de Escolas da Guia}, Guia, Portugal: \\
& \textit{Final grade: 18 out of 20.}\\
&\\
\end{tabular}

%----------------------------------------------------
%	SCHOLARSHIPS AND ADDITIONAL INFO
%----------------------------------------------------

\section{Scholarships and Certificates and Awards}

\begin{tabular}{rl}
\textsc{2014} & Contribution to the extra-curricular mobile Android app ActUA \normalsize\\
\textsc{2013} & First place on a University's Mobile App Development challenge awarded by Blip.pt \normalsize\\
\textsc{2011} & Research Integration Scholarship (12-month) at IEETA, financed by FCT \normalsize\\
\textsc{2008} & Top High School finalist. \normalsize\\
\end{tabular} \\

%----------------------------------------------------
%	COMPUTER SKILLS 
%----------------------------------------------------

\section{Computer Skills}

\begin{tabular}{rl}
Basic Experience:
& OpenDaylight, OpenFlow, Ruby, C, C++, ASP.NET, PHP, JavaScript, Objective-C, \\
& iOS, bison, MIPS assembly, x86 assembly, Dansguardian, Squid3, Jekyll\\
& \\
Intermediate Experience:
& OpenStack, Java, git, Python, bash, Open vSwitch, OpenWrt, Android, C\#, SQL\\
& HTML, Cisco IOS, GNU/Linux, Vim, {\LaTeX}, Jenkins, GlassFish, GNS3, UML, Excel\\
\end{tabular} \\

%----------------------------------------------------
%	WORK EXPERIENCE 
%----------------------------------------------------

\section{Work Experience}

\begin{tabular}{r|p{11cm}}
	\emph{Current} & Researcher and Software Engineer \\
    \textsc{October 2014} & Instituto de Telecomunicações \\
    & \footnotesize{Given the experience I acquired on OpenStack during my Master's Dissertation and the related work I did during it, I was invited to stay at Instituto de Telecomunicações doing research related to Network Functions Virtualization (NFV). There, I kept working on OpenStack although initially not upstream. Then started making contributions to the Group-Based Policy project for OpenStack. Most of my work, though, has been around other aspects of NFV: automating configuration of Virtual Network Functions (VNFs), improving and discussing the Traffic Steering implementation to meet the purposes of Service Function Chaining (SFC), testing and integrating other implementation artifacts of the team. }\\
	\multicolumn{2}{c}{} \\
	
	\emph{Current} & Developer, Designer and Marketeer \\
	\textsc{May 2013} & Wrkout \\ 
	& \footnotesize{Not really a job, but an amazing work experience. Wrkout is a mobile Android app which I've created. By dealing with multiple aspects related to developing, publishing and monetizing the app by myself, I have learned plenty. This project turned product also teaches me exactly why I should be working in a team.}\\
	\multicolumn{2}{c}{}\\
\end{tabular}

%----------------------------------------------------
%	LANGUAGES
%----------------------------------------------------

\section{Languages}

\begin{tabular}{rl}
	\textsc{English:} & Fluent\\
	
	\textsc{Portuguese:} & Mothertongue\\
	
	\textsc{Spanish:} & Basic Understanding\\
\end{tabular} \\

%----------------------------------------------------
%	INTERESTS AND ACTIVITIES
%----------------------------------------------------

\section{Interests and Activities}
Network Virtualization\\
Software\\
Software-Defined Networking\\
Cloud Computing\\
OpenStack\\
GNU/Linux\\
Open-Source\\
Technology, Gadgets\\
Brainstorming\\

%----------------------------------------------------
%       RESEARCH/PUBLICATIONS
%----------------------------------------------------

\section{Publications}
\begin{tabular}{rl}
\textsc{2015} & Cardoso, I., Barraca, J. P., Goncalves, C., Aguiar, R. L.: \\
& ``Seamless integration of Cloud and Fog networks''. \\
& 1st IEEE Conference on Network Softwarization (NetSoft 2015). \normalsize\\
\textsc{2014} & Paulo Dias, Tiago Sousa, Joao Parracho, Igor Cardoso, Andre Monteiro, Beatriz Sousa Santos: \\
& ``Student Projects Involving Novel Interaction with Large Displays''. \\
& IEEE Computer Graphics and Applications, vol. 34, no. 2, pp. 80-86, Mar.-Apr., 2014. \normalsize\\
\textsc{2014} & Tiago Sousa, Igor Cardoso, João Parracho, Paulo Dias, Beatriz Sousa Santos: \\
& ``DETI-Interact: Interaction with Large Displays in Public Spaces Using the Kinect''. \\
& HCI 2014 - 16th International Conference on Human-Computer Interaction: 196-206. \normalsize\\
\textsc{2012} & Cardoso, I., Dias, P., Sousa Santos, B.: \\
& ``Interaction with large displays in a public space using the Kinect sensor''. \\
& 20 Encontro português de Computação Gráfica - EPCG 2012, pp. 81–88 (2012).
\end{tabular} \\

%----------------------------------------------------
%	CONTRIBUTIONS
%----------------------------------------------------

\section{Contributions}
I have made upstream contributions to the Group-Based Policy project for OpenStack and the OpenStack Operations Guide. I have made other contributions to Open-Source software that are public but not (yet) upstream, including irssi, Yakuake and OpenStack. I have also developed other simple side projects in the past, some of them available at my GitHub. I have also made local contributions in the context of GLUA.

%----------------------------------------------------
%	PRESENCE
%
% This section will deal with both my presence on the web (how to reach me, by professional profiles, etc)
% as well as my presence regardings groups, associations, development teams, open source projects, etc.
%----------------------------------------------------

\section{Presence}
\begin{tabular}{r|p{11cm}}
    \emph{ATNoG} & Advanced Telecommunications and Networks Group \\
    \textsc{September 2013} & \footnotesize{I became associated with ATNoG as part of my Master's Thesis. This group is established inside Aveiro's pole of Instituto de Telecomunicações.}\\
    \multicolumn{2}{c}{} \\
    \emph{GLUA} & University of Aveiro's Linux Group \\
    \textsc{July 2011} & \footnotesize{Member of the University of Aveiro's Linux Group (Grupo Linux da Universidade de Aveiro).}\\
    \multicolumn{2}{c}{}\\
\end{tabular}

%----------------------------------------------------
%	EVENTS
%----------------------------------------------------

\section{Events}
\begin{tabular}{r|p{11cm}}
    \emph{As an attendee and meeting participant} & OpenStack Summit 2015, Vancouver \\
    \emph{As an attendee} & OpenStack Summit 2014, Paris \\
    \emph{As different roles} & Other events, conferences and challenges less related. \\
\end{tabular} \\

\end{document}
